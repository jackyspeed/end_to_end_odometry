\documentclass[]{article}

\usepackage{amsmath,amssymb,amstext} % Lots of math symbols and environments
\usepackage[margin=0.5in]{geometry}

%opening
\title{End to End EKF Formulation}
\author{Benjamin Skikos}

\begin{document}

\maketitle

\begin{abstract}
This is the formulation for the EKF layer. It is designed to fuse IMU information in using a simple, discrete-time, dynamic model. The IMU measurements are treated as inputs to the process model.
\end{abstract}

\section{States}

\begin{itemize}
	\item $p(t)$ Incremental displacement between previous and current timestep
	\item $v(t)$ Body frame velocity with respect to an inertial frame
	\item $\Theta_{global}(t)$ Orientation with respect to gravity parameterized as pitch and yaw Euler angles 
	\item $b_a(t)$ Accelerometer bias
	\item $\Theta_{inc}(t)$ Incremental rotation between previous and current timestep, parameterized and yaw, pitch, roll Euler angles.
	\item $b_g(t)$ Gyroscope bias
\end{itemize}

\section{Inputs}

Inputs are assumed constant from time $t_{i-1}$ to time $t_i$.

\begin{itemize}
	\item $a(t) = \bar{a}(t) + \eta_{acc}$ Acceleration as measured by the accelerometer. Modeled with white noise.
	\item $w(t) = \bar{w}(t) + \eta_{gyro}$ Angular velocity as measured by the gyroscope. Modeled with white noise.
\end{itemize}

\section{Process Model}

$R(\Theta)$ Is the function that maps YPR Euler angles to a rotation matrix. Any quantities denoted by $\eta$ are white noise.

\begin{align}
\delta &= t_i - t_{i-1} \\
p(t_i) &= R(\delta (\bar{w}(t_i) + \eta_{gyro} - b_g(t_{i-1})))\delta v(t_{i-1}) + \frac{\delta^2}{2}(R(\Theta_{global}(t_{i-1}) + \delta(\bar{w}(t_i) + \eta_{gyro} - b_g(t_{i-1})))g + \bar{a}(t_i) + \eta_{acc} - b_a(t_{i-1})) \\
v(t_i) &= R(\delta (\bar{w}(t_i) + \eta_{gyro} - b_g(t_{i-1}))) v(t_{i-1}) + \delta(R(\Theta_{global}(t_{i-1}) + \delta(\bar{w}(t_i) + \eta_{gyro} - b_g(t_{i-1})))g + \bar{a}(t_i) + \eta_{acc} - b_a(t_{i-1})) \\
\Theta_{global}(t_i) &= \Theta_{global}(t_{i-1}) + \delta(\bar{w}(t_i) + \eta_{gyro} - b_g(t_{i-1})) \\
b_a(t_i) &= b_a(t_{i - 1}) + \eta_{accbias} \\
\Theta_{inc}(t) &= \delta (\bar{w}(t_i) + \eta_{gyro} - b_g(t_{i-1})) \\
b_g(t_i) &= b_g(t_{i-1}) + \eta_{gyrobias}
\end{align}

The noise model for this process is non-additive due to the noise associated with the IMU inputs. Let $x(t)$ be the complete state, and $u(t)$ the complete input. Then the noise can be approximated as additive:

\begin{align}
\Sigma_{process} \sim \frac{\partial x(t)}{\partial u(t)} \Sigma_{input} \frac{\partial x(t)}{\partial u(t)}^T + \Sigma_{bias}
\end{align}

\section{Measurement Model}

The neural network is trained to directly output incremental position and orientation (with a YPR Euler angle parameterization). Therefore, the measurement model is simply the corresponding states.

\end{document}
